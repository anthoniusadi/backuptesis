\chapter{PENDAHULUAN}
\section{Latar Belakang Masalah}
Disabilitas adalah kelompok masyarakat yang memiliki keterbatasan yang dapat menghambat partisipasi dan peran serta mereka dalam kehidupan bermasyarakat. Penyandang disabilitas memiliki berbagai kategori, yaitu disabilitas fisik, intelektual, mental dan sensorik.
menurut data dari kemenkes tahun 2015, presentase 3 teratas penyandang disabilitas di provinsi indonesia adalah 6.36\% kesulitan melihat, 3.76\% kesulitan berjalan dan 3.35\% kesulitan mendengar (Kemenkes., 2018). Pemerintah Indonesia telah menandatangani konvensi tentang Hak-Hak Penyandang Disabilitas pada tanggal 30 Maret 2007 di New York. Adanya penandatanganan tersebut menunjukan bangsa indonesia menghormati, melindungi, memenuhi dan memajukan hak-hak penyandang disabilitas. Untuk itu perlu adanya dukungan dari masyarakat dalam mewujudkanya. 

Bagi penyandang disabilitas, mereka memiliki hambatan akses dalam melakukan aktivitas sehari hari. Dengan adanya perkembangan pengetahuan dan teknologi, mereka mulai terbantu dan dapat melakukan aktivitas layaknya masyarakat pada umumnya.
Mulai banyak penyandang disabilitas yang melakukan mobilitas tinggi dengan kursi roda, mengakses informasi dengan adanya penerjemah bahasa. Dengan adanya hal tersebut, penyandang disabilitas mendapat tempat dan peranan yang sama dengan masyarakat lainnya.

Teknologi yang semakin dewasa membuat mobilitas penyandang disabilitas menjadi lebih tinggi. Kursi roda mungkin dapat digunakan untuk membantu penyandang bergerak dari suatu tempat ke tempat lain, namun bagi beberapa penyandang disabilitas tertentu yang tidak memiliki kemampuan normal pada kondisi tangan atau lumpuh sebagian tidak dapat menggunakan kursi roda tersebut. Alhasil perlu adanya orang lain untuk membantu menngerakan kursi roda tersebut. 

Teknologi komputer dan robotika saat ini memiliki peranan penting dalam membantu sebuah permasalahan dari mulai kegiatan industri hingga kegiatan masyarakat. (Posada-Gómez et al., 2007) Membuat kursi roda pintar dengan kontrol gestur tangan, namun pada penelitian tersebut memiliki kelemahan terhadap cahaya. Intensitas cahaya yang rendah membuat sistem kurang mampu mendeteksi kontrol dari gestur tangan, sehingga hanya dapat digunakan dalam keadaan cahaya terang. Penelitian tersebut menggunakan teknologi pemrosesan citra yang dikombinasikan dengan elektronika dan mekanika untuk pergerakan kursi roda.

Pemrosesan citra memiliki beberapa hal fundamental permasalahan diantaranya adalah proses perbaikan citra. Perbaikan citra digunakan untuk memperbaiki sebuah citra yang bermasalah agar informasi citra terlihat lebih jelas secara visual maupun perhitungan. Penggunaan \emph{image processing} sangat dibutuhkan untuk membantu menyelesaikan permasalahan dalam kehidupan sehari hari. Aplikasi dari implementasi \emph{image processing} beberapa diantaranya adalah pengenalan dan deteksi pada sebuah objek. Untuk menyelesaikan permasalahan tersebut perlu membuat sistem yang tahan terhadap kondisi cahaya berintensitas rendah.

Pengenalan dan deteksi sebuah objek memiliki ruang lingkup yang sangat luas untuk di kembangkan, pada penelitian ini berfokus pada gesture recognition.
Dalam pemrosesan citra algoritme pada gesture recognition dapat di implementasikan pada komunikasi non verbal ataupun suatu gerakan yang dapat membantu seseorang menyelesaikan permasalahan. Pola pada gesture dapat dikenali oleh seseorang dengan cara melihat gesture tersebut, hal yang sama terjadi pada kamera yang mengadopsi apa yang dilakukan mata manusia untuk mengenali sebuah objek. 
\emph{Hand gesture} salah satu contoh implementasi \emph{image processing} yang dapat dikembangkan untuk membantu seseorang menyelesaikan permasalahan dengan menggunakan gestur tangan. 

Salah satu faktor yang dapat menurunkan kualitas suatu citra yaitu pencahayaan dari sebuah citra, proses pengambilan citra dalam intensitas rendah akan menghasilkan citra yang buruk (Saputra., 2016).
Implementasi algoritme untuk meningkatkan kualitas citra terhadap kondisi cahaya merupakan permasalahan yang menarik untuk diteliti.
Kondisi cahaya pada image processing adalah sesuatu topik yang menantang dalam permasalahan \emph{image processing} dan \emph{computer vision} untuk meningkatkan visibilitas maupun kualitas yang lebih baik dari suatu citra. 
Beberapa penelitian terdahulu telah melakukan peningkatan algoritme pada kondisi cahaya yang minim untuk suatu citra (Loh et al., 2019). 
Berkembangnya algoritme dalam ruang lingkup kondisi cahaya yang minim bertujuan mendapatkan kualitas kontras yang lebih jelas untuk dapat dilakukan komputasi lebih lanjut.

\emph{Retinex} merupakan metode yang diusulkan oleh Land dengan memodelkan pencahayaan dan persepsi warna berdasarkan penglihatan mata manusia. Mata manusia dapat membedakan sebuah objek sekalipun dalam kondisi intensitas cahaya yang rendah. Metode \emph{Retinex} terus mengalami berbagai pengembangan dari \emph{Single Scale Retinex} hingga \emph{Multiscale Retinex} berupaya untuk memperoleh keseimbangan kontras dalam pencahayaan berintensitas rendah(Saputra., 2016).

Peningkatan kontras menjadi salah satu solusi untuk meningkatkan visibilitas dari sebuah citra. Penelitian ini menggunakan algoritme \emph{Multiscale Retinex Color Restoration} untuk meningkatkan kontras serta \emph{object detection} untuk melakukan segmentasi pada suatu citra, kemudian untuk melakukan pengenalan sebuah gesture yang telah dilakukan perbaikan kontras, akan dilanjutkan dengan metode \emph{Convolutional Neural Network}. 

Implementasi dari peningkatan kontras untuk pengenalan gestur tangan ini diharapkan dapat dibawa untuk menyelesaikan permasalahan pada sistem yang bergantung pada cahaya. Beberapa sistem yang dimaksud seperti kursi roda pintar, ataupun sistem lainnya seperti pengenalan wajah dan pengenalan gestur. 
\section{Rumusan Masalah}
Berdasarkan latar belakang, deteksi dan pengenalan sebuah objek sangat dipengaruhi oleh cahaya.
Intensitas cahaya yang rendah dapat menyebabkan rendahnya akurasi pengenalan gestur tangan yang berdampak pada performa sebuah sistem terutama saat sistem diterapkan pada kondisi minim cahaya.
\section{Batasan Masalah}

Penelitian ini memiliki batasan masalah yang bertujuan untuk tidak memperluas pokok bahasan. Batasan masalah yang digunakan dalam penelitian ini adalah sebagai berkut:
\begin{enumerate}
\item Acuan dataset menggunakan \emph{American Sign Language}, dengan sepuluh klasifikasi yaitu angka 0 hingga angka 9.
\item Data yang digunakan menggunakan dataset private dan dataset public.
\item Proses pengujian akan dilakukan dengan 3 subjek dengan warna kulit yang berbeda.
\item Penurunan intensitas cahaya dikurangi sebesar 50\% lux sebelumnya hingga nilai lux kurang dari 50 lux.
\end{enumerate}
\section{Tujuan Penelitian}
Penelitian yang dilakukan bertujuan untuk mengatasi permasalahan sistem yang bergantung pada pencahayaan dengan mengimplementasikan algoritme \emph{Retinex} pada sebuah sistem deteksi dan pengenalan gestur tangan, dengan tingkat intensitas cahaya yang bervariasi.

\section{Manfaat Penelitian}
Penelitian ini diharapkan dapat memberi manfaat berupa:
\begin{enumerate}
\item Memberikan fitur tambahan untuk sistem yang memiliki permasalahan dengan pencahayaan dalam intensitas cahaya rendah.
\item Meningkatkan akurasi pengenalan gestur tangan pada kondisi cahaya berintensitas rendah.
\end{enumerate}

